\section{Wahrscheinlichkeitsmaße auf $\pmb{\R}$}
Kontinuierliche Ergebnisse lassen sich nicht mehr durch eine abzählbare Anzahl an Versuchsausgängen beschreiben.
Wahrscheinlichkeiten kann man dann nur noch \enquote{gutartigen Mengen} zuordnen, u.a.:
\begin{itemize}
	\item Intervalle sind gutartig
	\item Komplemente gutartiger Mengen sind gutartig
	\item Abzählbare Vereinigungen gutartiger Mengen sind gutartig
\end{itemize}
Bezeichne nun mit $\AG\subseteq\PG(\Omega)$ das System aller \enquote{gutartigen} Mengen.

\paragraph{Definition: $\pmb{\sigma}$-Algebra}
Sei $\Omega\neq\emptyset$ ein beliebiger Grundraum. Eine Menge $\AG\subseteq\PG(\Omega)$ heißt $\pmb{\sigma}$\textbf{-Algebra} auf $\Omega$, falls:
\begin{itemize}
	\item $\emptyset,\Omega\in\AG$
	\item $A\in\AG\implies A^{\mathsf{C}}\in\AG$
	\item $A_n\in\AG\;\forall n\in\N\implies\bigcup\limits_{n\in\N}A_n\in\AG$
\end{itemize}
$(\Omega,\AG)$ heißt dann \textbf{messbarer Raum}.
Die Mengen $A\in\AG$ heißen \textbf{Ereignisse}.

\paragraph{Definition: Borel-$\pmb{\sigma}$-Algebra}
Die \textbf{Borel-$\pmb{\sigma}$-Algebra} $\BG$ auf $\R$ beschreibt die kleinste $\sigma$-Algebra, welche alle Intervalle $(a,b]$ für beliebige $a,b\in\R$ enthält.

\paragraph{Definition: Wahrscheinlichkeitsmaß und Wahrscheinlichkeitsraum}
Sei $(\Omega,\AG)$ ein Messraum mit Grundraum $\Omega\neq\emptyset$ und $\sigma$-Algebra $\AG$.
Eine Abbildung $\PP:\AG\rightarrow[0,1]$ heißt Wahrscheinlichkeitsmaß auf $(\Omega,\AG)$, falls
\begin{itemize}
	\item $\PP(\Omega)=1$
	\item $A_n\in\AG,n\in\N$, disjunkt $\implies\PP(\bigcup\limits_{n\in\N}A_n)=\sum\limits_{n\in\N}\PP(A_n)$ \qquad($\sigma$-Additivität)
\end{itemize}
($\Omega,\AG\PP$) heißt dann \textbf{Wahrscheinlichkeitsraum}.

\newpage
\paragraph{Sätze und Definitionen für allgemeine Wahrscheinlichkeitsräume}
Folgende Sätze und Definitionen übertragen sich sinngemäß, wobei als Ereignisse jeweils nur Mengen aus $\AG$ betrachtet werden:
\begin{itemize}
	\item \hyperref[rules]{Rechenregeln für diskrete Wahrscheinlichkeitsmaße}
	\item \hyperref[conditioned]{Bedingte Wahrscheinlichkeiten}
	\item \hyperref[bayes]{Satz von der Totalen Wahrscheinlichkeit und von Bayes}
	\item \hyperref[independant]{Stochastische Unabhängigkeit von Ereignissen}
\end{itemize}
\underline{Unterschied}: Während diskrete Wahrscheinlichkeitsmaße $\PP$ vollständig durch die Zähldichte $f(\omega)\coloneqq\PP(\{\omega\}),\omega\in\Omega$ bestimmt sind, ist dies für allgemeine Wahrscheinlichkeitsmaße falsch!

\paragraph{Verteilungsfunktion}